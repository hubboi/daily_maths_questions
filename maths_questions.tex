\documentclass[12pt]{article}
% Packages %
\usepackage[margin=20mm]{geometry}
% margin setting
\usepackage{amsmath, amssymb}   % math formatting & symbols
\usepackage{graphicx}           % insert graphics
\usepackage{mwe}    % loads »blindtext« and »graphicx
\usepackage{fullpage}           % fullpage margins
\usepackage{esint}              % vector calculus
\usepackage{framed}             % frames section of doc
\usepackage{enumerate}          % allows custom numbers
\usepackage{enumitem}           % allows enumerating using anything
\usepackage[super]{nth}         % 1st, 2nd, 3rd etc...
\usepackage{amsfonts}           % needed for the overbar command
\usepackage{mathabx}            % astronomical symbols
\usepackage{tikz}               % for drawing diagrams 
\usepackage{caption}            % removes figure captions
\usepackage{subcaption}         % allows subfigure environment
\usepackage{mdwlist}            % allows continuation of list with suspend{enumerate} and resume{enumerate}
\usepackage{mathabx}            % contains planet symbols
\usepackage{graphics, setspace} % for mathematica code
\usepackage{slashed}            % feynman slash notation. Use \slashed
\usepackage{simplewick}         % wick contractions
\usepackage{pgfplots, pgfkeys}  % for tikz pictures
\usepackage{cancel}				% this allows the use of slash indicating cancellation \cancel{}
\usepackage{color}
\usepackage{hyperref}
\hypersetup{
    colorlinks,
    citecolor=black,
    filecolor=black,
    linkcolor=black,
    urlcolor=black
}

\let\OLDthebibliography\thebibliography
\renewcommand\thebibliography[1]{
  \OLDthebibliography{#1}
  \setlength{\parskip}{0pt}
  \setlength{\itemsep}{0pt plus 0.3ex}
} % change the spacing in between references

%Extras for tikz picture %
\usetikzlibrary{positioning,arrows,patterns}
\usetikzlibrary{decorations.markings}
\usetikzlibrary{calc} 

\definecolor{jblue}  {RGB}{20,50,100}
\definecolor{npurple}  {RGB} {153, 51, 204}
\definecolor{wred}   {RGB}{217,0,56}
\definecolor{white}   {RGB}{255,255,255}

\definecolor{korange}   {RGB}{235, 80,  43}
\definecolor{korange2}   {RGB}{245, 100,  63}
\definecolor{kyelloworange}   {RGB}{255, 210,  110}
\definecolor{kyelloworange2}   {RGB}{240, 170,  90}
\definecolor{kred}   {RGB}{204,  102, 153}
\definecolor{kpurple}   {RGB}{153,  61, 190}
\definecolor{kpurplelight}   {RGB}{213,  161, 230}

% Define styles for the different kind of edges in a Feynman diagram
\tikzset{
  phi/.style={dashed,draw=wred},
  antifermion/.style={draw=jblue,postaction={decorate},decoration={markings,mark=at position .55 with {\arrow[draw=jblue]{\langle }}}},
  fermion/.style={draw=jblue,postaction={decorate},decoration={markings,mark=at position .55 with {\arrow[draw=jblue]{\langle }}}},
  %gluon/.style={decorate,draw=magenta,decoration={coil,amplitude=4pt,segment length=5pt}}
  vertex/.style={draw,shape=circle,fill=black,minimum size=3pt,inner sep=0pt}
}

% Packages for Greek %
%\usepackage[LGRx,T1]{fontenc}
\usepackage[utf8]{inputenc}
\usepackage[greek, english]{babel}
\usepackage{fancyhdr}

% Fonts %
% \usepackage[condensed,math]{kurier}
% \usepackage[adobe-utopia]{mathdesign}
% \usepackage{eulervm, bookman}        

% \usepackage{cmbright}
% \usepackage[T1]{fontenc}

% Quantum Commands %
\newcommand{\bra}[1]{\left<  #1 \right|}
\newcommand{\ket}[1]{\left| #1 \right> }
\newcommand{\braket}[2]{\left<  #1 \middle| #2 \right> }
\newcommand{\matrixelement}[3]{\langle #1 | #2 | #3 \rangle}
\newcommand{\lagra}{\mathcal{L}}
\newcommand{\overbar}[1]{\mkern 1.5mu\overline{\mkern-1.5mu#1\mkern-1.5mu}\mkern 1.5mu}

\numberwithin{equation}{subsection}
% number the equation under according to sections
 
\pagestyle{fancy}
\fancyhf{}
\renewcommand{\headrulewidth}{1pt}
\fancyhead[LO,RE]{\rightmark}
\fancyhead[LE,RO]{\thepage}
%\lhead{} % left header
\setlength{\headsep}{0.3in}
%E for even page
%O for odd page
%L for left side
%C for centered
%R for right side
% \leftmark adds name and number of the current top-level structure (for example, Chapter for reports and books classes; Section for articles ) in uppercase letters.
% \rightmark adds name and number of the current next to top-level structure (Section for reports and books; Subsection for articles) in uppercase letters.

\begin{document}

\begin{titlepage}
    \begin{center}
        \vspace*{1cm}
        
        \textbf{\LARGE Daily Maths Questions}
        
        \vspace{1.5cm}
        
        \textbf{Ieng-Duan}
        
        \vspace{0.8cm}
        
        Lobb Street\\
        Brunswick\\
        Australia
        
    \end{center}
\end{titlepage}

\newpage
\section*{Question 1}

\section*{Question 2}


\section*{Question 3}
Find the indefinite integral of the following
\begin{align}
\int e^{e^{2016x} + 6048x} dx
\end{align}
\textbf{Solution}\\
It's important to realise that $6048 = 3\times 2016$. Let $u = 2016x$, then we have
\begin{align*}
\int e^{e^{2016x} + 6048x} dx & = \int e^{e^{u}+3u} du \frac{1}{2016}\\
&= \frac{1}{2016}\int e^{e^u}(e^{u})^3du. 
\end{align*}
Let $w = e^u$, we have
\begin{align*}
 \frac{1}{2016}\int e^{e^u}(e^{u})^3du &=  \frac{1}{2016}\int e^{w}w^3dw\frac{du}{dw}\\
&=  \frac{1}{2016}\int e^{w}w^3dw\frac{1}{\frac{dw}{du}}\\
&=  \frac{1}{2016}\int e^{w}w^3dw\frac{1}{w}\\
&=  \frac{1}{2016}\int e^{w}w^2dw\\
&=  \frac{1}{2016}(e^w w^2 - 2\int e^{w}wdw )\\
&=  \frac{1}{2016}(e^w w^2 - 2[ e^{w}w - \int e^wdw ])\\
&=  \frac{1}{2016}(e^w w^2 - 2e^{w}w +  e^w)\\
&=  \frac{1}{2016}(e^{e^u} (e^u)^2 - 2e^{e^u}e^u +  e^{e^u})\\
&=  \frac{1}{2016}(e^{e^{2016x}} (e^{2016x})^2 - 2e^{e^{2016x}}e^{2016x} +  e^{e^{2016x}})\\
&=  \frac{1}{2016}(e^{e^{2016x} + 4032x} - 2e^{e^{2016x} + 2016x} +  e^{e^{2016x}}).
\end{align*}

\section*{Question 4}
Find the indefinite integral of the following
\begin{align}
\int \frac{x^{-\frac{1}{2}}}{1+x^{\frac{1}{3}}} dx
\end{align}
\textbf{Solution}\\
The idea is to substitute $x$ with another variable that will make the new variable have integer powers. A common multiple of 2 and 3 is 6, hence let $u = x^{\frac{1}{6}}$, then $x = u^6$, and we have 
\begin{align*}
\int \frac{x^{-\frac{1}{2}}}{1+x^{\frac{1}{3}}} dx &= \int \frac{u^{-3}}{1+u^{2}} du\frac{dx}{du}\\
&= \int \frac{u^{-3}}{1+u^{2}}6 u^5du\\
&= 6\int \frac{u^{2}}{1+u^{2}} du\\
&= 6\int 1 - \frac{1}{1+u^{2}} du\\
&= 6u - 6\tan^{-1}(u) + C\\
&= 6x^{\frac{1}{6}} - 6\tan^{-1}(x^{\frac{1}{6}}) + C.
\end{align*}

\end{document} 
